%-------------------------------------------------------------------------------
\chapter{Introduction}
\label{ch.introduction}
%-------------------------------------------------------------------------------

%%%%%%%%%%%%%%%%%%%%%%%%%%%%%%%%%%%%%%%%%%%%%%%%%%%%%%%%%%%%%%%%%%%%%%%%%%%%%%%%
%%%%%%%%%%%%%%%%%%%%%%%%%%%%%%%%%%%%%%%%%%%%%%%%%%%%%%%%%%%%%%%%%%%%%%%%%%%%%%%%
%%%%%%%%%%%%%%%%%%%%%%%%%%%%%%%%%%%%%%%%%%%%%%%%%%%%%%%%%%%%%%%%%%%%%%%%%%%%%%%%
\section{Motivation}
Creation of artificial humans has been a prominent idea in many cultures across the
world \cite{KempHumanoids}. However, even replication of the simplest functions of human
brain and body is difficult, if possible, with the current development of technologies.
Humanoid robotics tries to develop robots that are capable of working in the 
environments adapted for humans, side by side with them or instead of them.
It is no wonder, that complexity and diversity of these problems are appealing for 
many researchers. This thesis is focused only on one aspect of development of humanoid 
robots -- on their locomotion.

Obviously, humanoid robots have to use their legs for locomotion. Modern research in 
this field begins in the $70$'s with the projects of I. Kato in Japan at Waseda University 
and M. Vukobratovi\'c at Mihalo Pupin Institute, Belgrade. The first group worked on 
anthropomorphic robots, while the second one -- on active exoskeletons for rehabilitation.
In spite of the progress made since then, many problems still do not have practical 
solutions. One of the most important restricting factors is complexity and intrinsic
nonlinearity of models of humanoid robots. It makes the control of such robots challenging
in real-time on the available computing hardware. Though explicit control is not always
necessary (for example, for passive walkers, which are briefly described in \cref{ch.background}),
it usually gives more flexibility. Hence, it is often necessary to employ approximate models, 
as well as highly optimized software to realize motions of humanoids. The goal of the
thesis is to design, implement and test on a humanoid platform an embedded optimization 
based control scheme for walking.


%%%%%%%%%%%%%%%%%%%%%%%%%%%%%%%%%%%%%%%%%%%%%%%%%%%%%%%%%%%%%%%%%%%%%%%%%%%%%%%%
%%%%%%%%%%%%%%%%%%%%%%%%%%%%%%%%%%%%%%%%%%%%%%%%%%%%%%%%%%%%%%%%%%%%%%%%%%%%%%%%
%%%%%%%%%%%%%%%%%%%%%%%%%%%%%%%%%%%%%%%%%%%%%%%%%%%%%%%%%%%%%%%%%%%%%%%%%%%%%%%%
\section{Contribution}
Apart from the work of many researchers in the field of legged locomotion this
thesis is a successor to the project of Antonio Paolillo \cite{AntonioThesis}. 
The main contribution of this thesis is development of an optimized solver for the 
quadratic problem (refer to \cref{ch.QP}), which is used for walking motion 
generation. The solver exploits the structure of the sparse formulation of the 
quadratic problem in order to achieve higher performance. The design of this 
solver and its performance are discussed in

\begin{itemize}
    \item D.~Dimitrov, A.~Sherikov, and P.B. Wieber.
        \newblock A sparse model predictive control formulation for walking motion
          generation.
        \newblock In {\em Intelligent Robots and Systems (IROS), 2011 IEEE/RSJ
          International Conference on}, pages 2292--2299. IEEE, 2011.
\end{itemize}

Furthermore, the software module for control of a Nao robot (see \cref{sec.nao})
developed in \cite{AntonioThesis} was rewritten and extended. The new version 
implements closed loop control with error feedback and uses the position of the 
center of mass for control decisions instead of position of the torso (refer to 
\cref{ch.naomodule} for more information).



%%%%%%%%%%%%%%%%%%%%%%%%%%%%%%%%%%%%%%%%%%%%%%%%%%%%%%%%%%%%%%%%%%%%%%%%%%%%%%%%
%%%%%%%%%%%%%%%%%%%%%%%%%%%%%%%%%%%%%%%%%%%%%%%%%%%%%%%%%%%%%%%%%%%%%%%%%%%%%%%%
%%%%%%%%%%%%%%%%%%%%%%%%%%%%%%%%%%%%%%%%%%%%%%%%%%%%%%%%%%%%%%%%%%%%%%%%%%%%%%%%
\section{Outline}
The thesis consists of seven chapters including this introductory chapter. 
\cref{ch.background} introduces terminology, describes basic concepts and reviews 
related works. The formulation of model predictive control problem for walking 
pattern generation and some general notes on the model predictive control are 
given in \cref{ch.MPC}. The next \cref{ch.QP} discusses implementation of 
optimized solvers for the forenamed model predictive control problem. The design 
of software walking module for the Nao robot and the results of the experiments 
on the robot and in a simulation are presented in \cref{ch.naomodule} and 
\cref{ch.results}, respectively. The results are summarized in \cref{ch.conclusion},
which also discusses possible future work. \cref{app.Schur} contains sample
derivations of the Schur complement, which is used to solve the model predictive
control problem.


%%%%%%%%%%%%%%%%%%%%%%%%%%%%%%%%%%%%%%%%%%%%%%%%%%%%%%%%%%%%%%%%%%%%%%%%%%%%%%%%
%%%%%%%%%%%%%%%%%%%%%%%%%%%%%%%%%%%%%%%%%%%%%%%%%%%%%%%%%%%%%%%%%%%%%%%%%%%%%%%%
%%%%%%%%%%%%%%%%%%%%%%%%%%%%%%%%%%%%%%%%%%%%%%%%%%%%%%%%%%%%%%%%%%%%%%%%%%%%%%%%
\section{Notation}
Names of programs and software libraries, names of constants, variables and 
functions that are used in programs are typed in monospace font, for example, \verb|Eigen|.

Matrices are denoted by bold capital letters, vectors -- by bold letters in lower
case, scalars -- by letters in lower case, for example, $\mbm{C}, \mbm{x}, y$.

Quadratic forms in mathematical expressions are denoted as
$$
\norm{\mbm{x}}^2_{\mbm{Q}} \triangleq \mbm{x}^T \mbm{Q} \mbm{x}.
$$

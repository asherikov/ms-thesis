%-------------------------------------------------------------------------------
\chapter{Derivation of Schur Complement}
\label{app.Schur}
%-------------------------------------------------------------------------------
Here an example derivation of the Schur complement of the \ac{KKT} 
system~\eqref{eq.kkt} is presented.

The Schur complement in \cref{sec.schur} was derived as
\begin{align*}
\mbm{S} = \frac{1}{2}\tilde{\mbm{E}}_c\tilde{\mbm{H}}_c^{-1}\tilde{\mbm{E}}_c^T 
+ \frac{1}{2}\tilde{\mbm{E}}_u\mbm{H}_u^{-1}\tilde{\mbm{E}}_u^T.
\end{align*}

When $N = 3$ the two terms in the summation are
\begin{align*}
\tilde{\mbm{E}}_c\tilde{\mbm{H}}_c^{-1}\tilde{\mbm{E}}_c^T 
&= 
    \begin{bmatrix} 
      -\mbm{I}    &  \mbm{0}    &  \mbm{0}         \\
       \tilde{\mbm{A}}_1 & -\mbm{I}    &  \mbm{0}  \\
       \mbm{0}    &  \tilde{\mbm{A}}_2 & -\mbm{I}  \\
    \end{bmatrix}
    \begin{bmatrix} 
      \tilde{\mbm{Q}} & \mbm{0}    & \mbm{0}    \\
      \mbm{0}    & \tilde{\mbm{Q}} & \mbm{0}    \\
      \mbm{0}    & \mbm{0}    & \tilde{\mbm{Q}} \\
    \end{bmatrix}^{-1}
    \begin{bmatrix} 
      -\mbm{I} &  \tilde{\mbm{A}}^T_1 &  \mbm{0}             \\
       \mbm{0} & -\mbm{I}             &  \tilde{\mbm{A}}^T_2 \\
       \mbm{0} &  \mbm{0}             & -\mbm{I}             \\
    \end{bmatrix}\\
&= 
  \begin{bmatrix} 
    \tilde{\mbm{Q}}^{-1}    &  -\tilde{\mbm{Q}}^{-1}\tilde{\mbm{A}}^T_1    &  \mbm{0}\\
    -\tilde{\mbm{A}}_1\tilde{\mbm{Q}}^{-1} & \tilde{\mbm{A}}_1\tilde{\mbm{Q}}^{-1}\tilde{\mbm{A}}^T_1 + \tilde{\mbm{Q}}^{-1}   &  -\tilde{\mbm{Q}}^{-1}\tilde{\mbm{A}}^T_2 \\
    \mbm{0}    &  -\tilde{\mbm{A}}_2\tilde{\mbm{Q}}^{-1} & \tilde{\mbm{A}}_2\tilde{\mbm{Q}}^{-1}\tilde{\mbm{A}}^T_2 + \tilde{\mbm{Q}}^{-1}\\
  \end{bmatrix},\\
%
%
\tilde{\mbm{E}}_u\mbm{H}_u^{-1}\tilde{\mbm{E}}_u^T 
&=
    \begin{bmatrix} 
      \tilde{\mbm{B}}_0 & \mbm{0}    & \mbm{0}   \\
      \mbm{0}    & \tilde{\mbm{B}}_1 & \mbm{0}   \\
      \mbm{0}    & \mbm{0}    & \tilde{\mbm{B}}_2\\
    \end{bmatrix}
  \begin{bmatrix} 
    \mbm{P} & \mbm{0}    & \mbm{0}   \\
    \mbm{0}    & \mbm{P} & \mbm{0}   \\
    \mbm{0}    & \mbm{0}    & \mbm{P}\\
  \end{bmatrix}^{-1}
    \begin{bmatrix} 
      \tilde{\mbm{B}}^T_0 & \mbm{0}    & \mbm{0}    \\
      \mbm{0}    & \tilde{\mbm{B}}^T_1 & \mbm{0}    \\
      \mbm{0}    & \mbm{0}    & \tilde{\mbm{B}}^T_2 \\
    \end{bmatrix}\\
&=
  \begin{bmatrix} 
    \tilde{\mbm{P}}_0 & \mbm{0}    & \mbm{0}    \\
    \mbm{0}    & \tilde{\mbm{P}}_1 & \mbm{0}    \\
    \mbm{0}    & \mbm{0}    & \tilde{\mbm{P}}_2 \\
  \end{bmatrix}, 
\end{align*}
where $\tilde{\mbm{P}}_k = \tilde{\mbm{B}}_k\mbm{P}^{-1}\tilde{\mbm{B}}^T_k$.

Hence, $\mbm{S}$ has a block-diagonal structure with the following blocks
\begin{equation*}
\begin{split}
  2\mbm{S}_{11} &= \tilde{\mbm{Q}}^{-1} + \tilde{\mbm{P}}_0  \\
  2\mbm{S}_{kk} &= \mbm{A}_{k-1}\tilde{\mbm{Q}}^{-1}\mbm{A}^T_{k-1} + \tilde{\mbm{Q}}^{-1} + \tilde{\mbm{P}}_{k-1}  \\
  2\mbm{S}_{k,k+1} &= \mbm{S}_{k+1,k}^T = -\tilde{\mbm{Q}}^{-1}\mbm{A}^T_{k}.
\end{split}
\end{equation*}

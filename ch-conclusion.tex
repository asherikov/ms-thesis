%-------------------------------------------------------------------------------
\chapter{Conclusion}
\label{ch.conclusion}
%-------------------------------------------------------------------------------
A short summary of the work presented in the thesis is given in this chapter. This
work includes development of walking control scheme, its tuning and testing, and 
discussion of associated issues.

Several software components were developed in order to test the control scheme on a
robot:
\begin{itemize}
    \item the sparse active set and logarithmic barrier solvers for \ac{MPCWMG}
        (\cref{ch.QP,ch.naomodule}); 
    \item the inverse kinematics library (\cref{ch.naomodule});
    \item the footstep pattern generator (\cref{ch.naomodule});
    \item the middleware (\cref{ch.naomodule}), which integrates other components 
        and interacts with the Nao robot.
\end{itemize}

In order to obtain an admissible gait, a number of experiments were conducted using
the walking module (\cref{ch.results}). The experiments allowed to select an appropriate 
version of inverse kinematics, find good trajectories for the swing foot, and determine
parameters of the \ac{MPCWMG} as well as parameters of the solvers. 

The performance of the active set solver based on the sparse formulation compared 
to the dense formulation was presented in \cite{dimitrov2011sparse}. Though, the 
sparse formulation may improve performance, its implementation and extension (for 
example, to perform footstep repositioning \cite{dimitrov2011walking}) requires 
more effort.

The sparse active set and logarithmic barrier solvers were compared on a robot and 
in a simulation under disturbances (\cref{ch.results}). The results demonstrated, 
that the logarithmic barrier method is sufficiently fast for the real-time operation 
on a robot, and may be preferable in some situations.

Some heuristics were also tested:
\begin{itemize} 
    \item Approximation of the \ac{DS} with a sequence of rectangles 
        (\cref{sec.ds_constraints}) works reasonably well, however, the narrower 
        rectangular constraints may lead to a higher number of active constraints at 
        the optimal point. 
    \item Variation of the duration of the sampling periods in the preview window 
        (\cref{sec.timevarsys,sec.parameters}) gives more freedom for tuning, 
        but distribution of the \ac{DS} and \ac{SS} between different sampling 
        periods is problematic. 
    \item Another heuristic is employed to account for the computational delay of 
        the control loop (\cref{sec.mpc_delay}).
\end{itemize}

The rest of the chapter is devoted to discussion of the possible future work.



%%%%%%%%%%%%%%%%%%%%%%%%%%%%%%%%%%%%%%%%%%%%%%%%%%%%%%%%%%%%%%%%%%%%%%%%%%%%%%%%
%%%%%%%%%%%%%%%%%%%%%%%%%%%%%%%%%%%%%%%%%%%%%%%%%%%%%%%%%%%%%%%%%%%%%%%%%%%%%%%%
%%%%%%%%%%%%%%%%%%%%%%%%%%%%%%%%%%%%%%%%%%%%%%%%%%%%%%%%%%%%%%%%%%%%%%%%%%%%%%%%
\section{Discussion of Future Work}
The possible extensions of the work presented in the thesis can be divided into
two groups. The first group is focused on the control scheme, while the second
one on its implementation.

The control scheme can be potentially improved in several ways. It is interesting
to investigate the effect of introduction of stabilizing constraints to the \ac{MPCWMG}
as described in \cref{sec.mpc_stability}. The joint limits can be explicitly accounted
for in the \ac{IK} problem (see \cref{sec.ik_alg}). However, the \ac{IK} problem with 
the inequality constraints would need a \ac{QP} solver. The \ac{IK} module
can be rewritten in order to take the position of torso as the base in the same way as
it is made in \cite{NaoWalk}. The precision of the \ac{IK} may be improved by exclusion 
of the support foot/feet from computation of the position of the \ac{CoM}. Note that the 
sparse \ac{MPCWMG} formulation allows easy change of the height of \ac{CoM} during 
walk. Exploitation of this feature may lead to a more natural gait. It is also possible 
to implement and test on a robot repositioning of footsteps under disturbances, which 
is described in \cite{dimitrov2011walking}.

The second group of extensions include various heuristics and fine adaption of the
walking module to the Nao platform. The available pressure sensors can be used to 
improve error feedback. The module can be modified to account for the sensor delay,
which is quite significant ($10$ millisecond). Inclination of the sole of the swing 
foot may improve walk on uneven terrain.

%TODO    \item \ac{ZMP} multibody

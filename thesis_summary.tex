\documentclass[a4paper,12pt]{article}

%\usepackage[koi8-r]{inputenc}
%\usepackage[english,russian]{babel}
\usepackage{graphicx}
%\usepackage[ruled,lined,nofillcomment]{algorithm2e}
\usepackage{longtable}
\usepackage{listings}
%\usepackage{latexsym}
\usepackage{textcomp}
\usepackage{color}
\usepackage{url}
\usepackage{hyperref}
\usepackage{afterpage}
\usepackage{placeins}
\usepackage{amsmath}
\usepackage{amssymb} %\mathbb{R}


\makeatletter

\renewcommand{\labelenumi}{\theenumi)}

\pagestyle{plain}
%\raggedbottom

\makeatletter
\makeatother

\renewcommand{\labelenumi}{\theenumi)}
%\renewcommand{\lstlistingname}{�������}

%\textheight=25.5cm
%\topskip=1cm
%\textwidth=15.5cm
%\voffset=-20mm
%\hoffset=-10mm
\oddsidemargin=0cm
\textwidth=15.5cm
\topmargin=-0.54cm
\textheight=25.5cm
\headheight=0cm
\headsep=0cm

\sloppy

\renewcommand{\baselinestretch}{1.1}



\begin{document}
\begin{center}
{\Large THESIS SUMMARY}\\
\vspace{0.2cm}
{\large Alexander Sherikov}\\
\vspace{0.2cm}
{\large Model predictive control of a walking bipedal robot using online optimization}\\
\end{center}

The inherent complexity of humanoid robots makes their control a difficult task even
on the modern hardware. In order to address this issue approximate models and
high performance algorithms are employed. The thesis is focused on the model
predictive control of a walking bipedal robot, which is approximated by an inverted
pendulum, using online optimization. The goal of the thesis is to design, implement 
and test on a humanoid platform an embedded optimization based control scheme for walking.
A special emphasis is made on the solvers that exploit the structure of quadratic 
optimization problems in the context of model predictive control. 

The thesis consists of seven chapters. Chapter~1 is devoted to a short introduction. 
Chapter~2 introduces terminology, describes basic concepts and reviews related works. 
The formulation of model predictive control problem for walking pattern generation 
and some general notes on the model predictive control are given in Chapter~3. Chapter~4 
discusses implementation of optimized solvers for the forenamed model predictive control 
problem. The design of software walking module for the Nao platform and the results of 
experiments on the robot and in a simulation are presented in Chapters~5 and~6, respectively. 
The results are summarized in Chapter~7, which also discusses possible future work.

The main contribution (Chapter~4) of this thesis is development of an optimized solver of 
the quadratic problem, which is used for walking motion generation. Two methods of
solution of this problem were implemented and compared in a simulation and on a humanoid 
robot: primal active set and primal logarithmic barrier methods. Both exploit the structure 
of the problem in order to achieve higher performance. 

A software walking module was developed (Chapter~5) in order to perform tests on a robot. 
Apart from the solver, the key components of the module are: an inverse kinematics library; 
a simple footstep pattern generator; and middleware, which integrates other components and 
interacts with a Nao robot. Experiments on a robot (Chapter~6) allowed to select an 
appropriate version of inverse kinematics, find good trajectories for the swing foot, and 
tune parameters of the solver. 

A number of heuristics were employed in the walking module and solver. Double support 
constraints are approximated to simplify implementation (Section 3.2). Variation of sampling 
time allows to increase time span of the preview window without increasing the size of 
respective quadratic problem (Section 3.3). A special heuristic was developed in order to 
account for the computational delay of the control loop (Section 5.4).

\end{document}

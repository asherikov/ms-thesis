\pdfbookmark[0]{Title}{title}
\maketitle



%%%%%%%%%%%%%%%%%%%%%%%%%%%%%%%%%%%%%%%%%%%%%%%%%%%%%%%%%%%%%%%%%%%%%%%%%%%%%%%%
%-------------------------------------------------------------------------------

\frontmatter



%%%%%%%%%%%%%%%%%%%%%%%%%%%%%%%%%%%%%%%%%%%%%%%%%%%%%%%%%%%%%%%%%%%%%%%%%%%%%%%%
%-------------------------------------------------------------------------------
\cleardoublepage
\phantomsection
\pdfbookmark[0]{Abstract}{abstract}
\chapter*{Abstract}
%-------------------------------------------------------------------------------
Humanoid robotics is a challenging and promising research field. Legged locomotion
is one of the most important aspects of it. In spite of the progress achieved in 
the last years in control of walking robots, many problems are yet to be resolved. 
The inherent complexity of such robots makes their control a difficult task even 
on the modern hardware. In order to address this issue approximate models and
high performance algorithms are employed. This thesis is focused on the model 
predictive control of a walking bipedal robot, which is approximated by an inverted 
pendulum, using online optimization. A special emphasis is made on the solvers that 
exploit the structure of quadratic optimization problems in the context of model 
predictive control. Two methods for solution of these problems are implemented: 
primal active set and primal logarithmic barrier methods. They are tested and 
compared in a simulation and on a humanoid robot. A software module for control 
of the Nao humanoid robot is developed for this purpose.



%%%%%%%%%%%%%%%%%%%%%%%%%%%%%%%%%%%%%%%%%%%%%%%%%%%%%%%%%%%%%%%%%%%%%%%%%%%%%%%%
%-------------------------------------------------------------------------------
\cleardoublepage
\phantomsection
\pdfbookmark[0]{Acknowledgements}{acknowledgements}
\chapter*{Acknowledgements}
%-------------------------------------------------------------------------------
First of all I want to thank my parents and sister, without their support and 
understanding this work would be impossible. Also, I would like to thank my
supervisor Dimitar Dimitrov, whose guidance and advice were indispensable.



%%%%%%%%%%%%%%%%%%%%%%%%%%%%%%%%%%%%%%%%%%%%%%%%%%%%%%%%%%%%%%%%%%%%%%%%%%%%%%%%
%-------------------------------------------------------------------------------
\cleardoublepage
\phantomsection
\pdfbookmark[0]{Contents}{contents}
\setcounter{tocdepth}{2}
\tableofcontents

\cleardoublepage
\phantomsection
\pdfbookmark[1]{List of Figures}{listoffigures}
\listoffigures

\cleardoublepage
\phantomsection
\pdfbookmark[1]{List of Tables}{listoftables}
\listoftables
%\lstlistoflistings

\cleardoublepage
\phantomsection
\pdfbookmark[1]{List of Algorithms}{listofalgorithms}
\listofalgorithms

\cleardoublepage
\phantomsection
\pdfbookmark[1]{List of Acronyms}{listofacronyms}
\chapter*{List of Acronyms}
\begin{acronym}[WWWWWWW]
\acro{3D-LIPM}{Three-dimensional Linear Inverted Pendulum Model}
\acro{API}{Application Programming Interface}
\acro{CPU}{Central Processing Unit}
\acro{CoM}{Center of Mass}
\acro{CoP}{Center of Pressure}
\acro{DCM}{Device Communication Manager}
\acro{DOF}{Degrees of Freedom}
\acro{DS}{Double Support}
\acro{FK}{Forward Kinematics}
\acro{FZMP}{Fictitious Zero Moment Point}
\acro{IGM}{Inverse Geometrical Model}
\acro{IK}{Inverse Kinematics}
\acro{KKT}{Karush-Kuhn-Tucker conditions}
\acro{LMPC}{Linear Model Predictive Control}
\acro{LQR}{Linear Quadratic Regulator}
\acro{MPC}{Model Predictive Control}
\acro{MPCWMG}{Model Predictive Control for Walking Motion Generation}
\acro{NMPC}{Nonlinear Model Predictive Control}
\acro{PoS}{Polygon of Support}
\acro{QP}{Quadratic Program}
\acro{SDK}{Software Development Kit}
\acro{SMPC}{Sparse Model Predictive Control}
\acro{SS}{Single Support}
\acro{WMG}{Walking Motion Generation}
\acro{ZMP}{Zero Moment Point}
\end{acronym}
